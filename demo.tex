% This is a simple sample document.  For more complicated documents take a look in the exercise tab. Note that everything that comes after a % symbol is treated as comment and ignored when the code is compiled.
% !TEX program = xelatex

\documentclass{ctexart} % \documentclass{} is the first command in any LaTeX code.  It is used to define what kind of document you are creating such as an article or a book, and begins the document preamble

\usepackage{amsmath} % \usepackage is a command that allows you to add functionality to your LaTeX code
\usepackage[T1]{fontenc}
\usepackage{lmodern}  % scalable CM/Latin Modern fonts
\usepackage{fix-cm}   % allow arbitrary scaling (reduces size substitution warnings

\title{Simple Sample} % Sets article title
\author{My Name} % Sets authors name
\date{\today} % Sets date for date compiled

% The preamble ends with the command \begin{document}
\begin{document} % All begin commands must be paired with an end command somewhere
\maketitle % creates title using information in preamble (title, author, date)

\section{Hello World!} % creates a section

\textbf{Hello World!} Today I am learning \LaTeX. %notice how the command will end at the first non-alphabet charecter such as the . after \LaTeX
\LaTeX{} is a great program for writing math. I can write in line math such as $a^2+b^2=c^2$ %$ tells LaTexX to compile as math
. I can also give equations their own space:
\begin{equation} % Creates an equation environment and is compiled as math
  \gamma^2+\theta^2=\omega^2
\end{equation}

这是一个物理公式:
\begin{equation}
  e=mc^2 
  \label{eq:energy}
\end{equation}

$e=mc^2$

$$a^2=b^2+c^2$$

如公式~\eqref{eq:energy}~所示,能量与质量成正比。
If I do not leave any blank lines \LaTeX{} will continue  this text without making it into a new paragraph.  Notice how there was no indentation in the text after equation (1).
Also notice how even though I hit enter after that sentence and here $\downarrow$

% 下面演示交叉引用:使用 \label 定义标签,使用 \ref 或 \eqref 调用。
参考公式~\eqref{eq:energy}~可知,质能可以互相转换。

\section{交叉引用示例}
我们再写一个勾股定理公式,并引用它:
\begin{equation}
  a^2 + b^2 = c^2 \label{eq:pythag}
\end{equation}
如公式~\eqref{eq:pythag}~所示,直角三角形两直角边平方和等于斜边平方。

再引用一次上面的爱因斯坦公式:公式编号是~\ref{eq:energy}。使用 \\eqref 会自动加上括号,比如 \eqref{eq:energy}。

让我自己引用一次我的爱因斯坦公式:公式编号是~\eqref{eq:energy}。
让我自己引用一次我的爱因斯坦公式:公式编号是~\eqref{eq:energy}。

常见导致出现 ?? 的原因:
\begin{itemize}
  \item 只编译了一次:需要至少运行两遍 (推荐使用 latexmk 自动多轮编译)。
  \item 用了 \\cite 而不是 \\ref \\eqref : \\cite 是引用文献条目的,需要配合 bib 文件。
  \item 标签拼写不一致,或复制时多/少了空格。
  \item 标签放在不可编号的环境里(如普通段落)。
  \item 删除了 .aux 但没重新完整编译。
\end{itemize}

本地建议编译命令(运行两遍或用 latexmk):
% xelatex demo.tex
% xelatex demo.tex
% 或者:latexmk -xelatex demo.tex

如果仍出现 ?? ,清理辅助文件再重新编译(不要删除源 .tex):
% latexmk -C
% latexmk -xelatex demo.tex

\LaTeX{} formats the sentence without any break.  Also look how it doesn't matter how many spaces I put between my words.

For a new essay I can leave a blank space in my code.

For a new essay I can leave a blank space in my code.

\section{Conclusion} % creates a new section\

\subsection{Subsection} % creates a subsection
\subsubsection{Subsubsection} % creates a subsubsection
\begin{itemize}
  \item 这是一个子项目。
  \item 这是一个子子项目。

\end{itemize}

\[\begin{cases}
  collaborate
\end{cases} \]
\[
f(x) = \begin{cases} 
x^2 & \text{如果 } x \geq 0 \\
-x & \text{如果 } x < 0 
\end{cases}
\]


edit together
现在,
\end{document} % This is the end of the document