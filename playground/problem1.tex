% This is a simple sample document.  For more complicated documents take a look in the exercise tab. Note that everything that comes after a % symbol is treated as comment and ignored when the code is compiled.
% !TEX program = xelatex

\documentclass{ctexart} % \documentclass{} is the first command in any LaTeX code.  It is used to define what kind of document you are creating such as an article or a book, and begins the document preamble

\usepackage{amsmath} % \usepackage is a command that allows you to add functionality to your LaTeX code

\title{Simple Sample} % Sets article title
\author{My Name} % Sets authors name
\date{\today} % Sets date for date compiled

% The preamble ends with the command \begin{document}
\begin{document} % All begin commands must be paired with an end command somewhere
\maketitle % creates title using information in preamble (title, author, date)

\section{平行平面薄膜干涉理论(deepseekv3.1)}
\subsection{干涉光路与光程差分析}
考虑如图1所示的干涉光路:在折射率为 \(n_1\) 的均匀介质中,有一厚度为 \(d\) 的平行平面薄膜,其折射率为 \(n_2\),且满足 \(n_1 < n_2\)。当一束单色平行光 \(S\) 以入射角 \(i\) 照射到薄膜表面时,在界面处发生反射和折射现象。

extbf{两条干涉光路的形成:}
\begin{itemize}
    \item 光线a:在薄膜上表面直接反射;
    \item 光线b:折射进入薄膜,在下界面反射后再折射返回。
\end{itemize}

这两束光源于同一光源,满足相干条件,经透镜会聚后产生干涉现象。

extbf{光程差计算:}
考虑半波损失(光从光疏介质射向光密介质时相位发生 \(\pi\) 突变),两束光的光程差为:
\[
    \delta = n_2(AB + BC) - n_1\, AD + \frac{\lambda}{2}.
\]
根据几何关系:
\[
    AB = BC = \frac{d}{\cos\gamma},\qquad
    AD = AC\, \sin i = 2d\, \tan\gamma\, \sin i.
\]
结合折射定律 \(n_1 \sin i = n_2 \sin\gamma\),整理得:
\[
    \delta = 2d\, \sqrt{n_2^2 - n_1^2 \sin^2 i} + \frac{\lambda}{2}.
\]
对于透射光情况,由于不涉及半波损失,光程差为:
\[
    \delta_{\text{透}} = 2d\, \sqrt{n_2^2 - n_1^2 \sin^2 i}.
\]

\paragraph{特殊情况——垂直入射(\(i = 0^\circ\)):}
\begin{itemize}
    \item 反射光光程差:\(\delta = 2n_2 d + \tfrac{\lambda}{2}\);
    \item 透射光光程差:\(\delta_{\text{透}} = 2n_2 d\)。
\end{itemize}

\subsection{反射率与折射率关系}
基于菲涅尔公式,我们分别推导不同情况下的反射率表达式。设空气折射率为 \(n_1\),外延层折射率为 \(n_2\),衬底折射率为 \(n_3\),入射角为 \(\theta_1\),在外延层中的折射角为 \(\theta_2\)。

\subsubsection{垂直入射情况(\(\theta_1 = 0^\circ\))}
对于垂直入射情况,可以直接在两个界面上应用反射率公式,从而分别得到如下公式。
\paragraph{空气-外延层界面(1-2界面):}
\[
    R_{12} = \left( \frac{n_1 - n_2}{n_1 + n_2} \right)^2.
\]
\paragraph{外延层-衬底界面(2-3界面):}
\[
    R_{23} = \left( \frac{n_2 - n_3}{n_2 + n_3} \right)^2.
\]

\subsubsection{斜入射情况(\(\theta_1 > 0^\circ\))}
\paragraph{s 偏振光(垂直偏振):}
\begin{align*}
    R_{s12} & = \left( \frac{n_1 \cos\theta_1 - n_2 \cos\theta_2}{n_1 \cos\theta_1 + n_2 \cos\theta_2} \right)^2, \\
            & \quad \cos\theta_2 = \sqrt{1 - \left( \frac{n_1}{n_2} \sin\theta_1 \right)^2};                      \\[4pt]
    R_{s23} & = \left( \frac{n_2 \cos\theta_2 - n_3 \cos\theta_3}{n_2 \cos\theta_2 + n_3 \cos\theta_3} \right)^2, \\
            & \quad \cos\theta_3 = \sqrt{1 - \left( \frac{n_2}{n_3} \sin\theta_2 \right)^2}.
\end{align*}

\paragraph{p 偏振光(平行偏振):}
\[
    R_{p12} = \left( \frac{n_1 \cos\theta_2 - n_2 \cos\theta_1}{n_1 \cos\theta_2 + n_2 \cos\theta_1} \right)^2,\qquad
    R_{p23} = \left( \frac{n_2 \cos\theta_3 - n_3 \cos\theta_2}{n_2 \cos\theta_3 + n_3 \cos\theta_2} \right)^2.
\]

\section{问题一的数学模型构建}

基于上述光学原理,我们建立碳化硅外延层厚度与测量参数的数学模型:

\subsection{基本参数定义}
\begin{itemize}
    \item 波数:\(q = 1/\lambda\)(单位:cm\(^{-1}\));
    \item 外延层折射率:\(n_2 = 2.55\)(基于文献值);
    \item 衬底折射率:\(n_3 = 2.55\)(假设与外延层一致);
    \item 空气折射率:\(n_1 = 1.0\);
    \item 入射角:\(i = 10^\circ\) 或 \(15^\circ\)(根据实验条件)。
\end{itemize}

\subsection{干涉极值条件}
反射光强出现极大值的条件为:\(\delta = m\lambda\ (m = 1,2,3,\ldots)\)。代入光程差表达式:
\[
    2d\, \sqrt{n_2^2 - n_1^2 \sin^2 i} + \frac{\lambda}{2} = m\lambda,
\]
整理得:
\[
    2d\, \sqrt{n_2^2 - n_1^2 \sin^2 i} = \left(m - \tfrac{1}{2}\right)\lambda.
\]

\subsection{厚度计算公式推导}
将波长转换为波数(\(\lambda = 1/q\)),得到:
\[
    2d\, \sqrt{n_2^2 - n_1^2 \sin^2 i} = \frac{m - \tfrac{1}{2}}{q}.
\]
对于相邻两个干涉极大值(对应波数 \(q_k\) 和 \(q_{k+1}\),干涉级次 \(m\) 和 \(m-1\)):
\[
    \begin{cases}
        2d\, \sqrt{n_2^2 - n_1^2 \sin^2 i} = \dfrac{m - \tfrac{1}{2}}{q_k}, \\
        2d\, \sqrt{n_2^2 - n_1^2 \sin^2 i} = \dfrac{(m-1) - \tfrac{1}{2}}{q_{k+1}}.
    \end{cases}
\]
更合理的方法是直接利用相邻极值的条件:由 \(2d \sqrt{n_2^2 - n_1^2 \sin^2 i} = \frac{m - \tfrac{1}{2}}{q_m}\),对于第 \(m\) 级和第 \(m+1\) 级干涉极大值:
\[
    2d \sqrt{n_2^2 - n_1^2 \sin^2 i} = \frac{m - \tfrac{1}{2}}{q_m} = \frac{(m+1) - \tfrac{1}{2}}{q_{m+1}}.
\]
由此可得厚度计算公式:
\[
    d = \frac{1}{2\sqrt{n_2^2 - n_1^2 \sin^2 i}}\, \cdot \frac{1}{q_m - q_{m+1}}.
\]
或者更一般地,对于任意两个相邻极值点(在 \(n_1=1\) 的情形下):
\[
    d = \frac{1}{2\sqrt{n_2^2 - \sin^2 i}}\, \cdot \frac{1}{\Delta q},\qquad \Delta q = \lvert q_k - q_{k+1} \rvert.
\]

\subsection{模型应用说明}
\begin{enumerate}
    \item 数据预处理:从实验数据中识别反射率极大值点;
    \item 参数确定:根据实验条件确定入射角 \(i\) 和折射率 \(n_2\);
    \item 厚度计算:利用相邻极值点波数差计算厚度;
    \item 结果验证:通过多个极值点对计算结果进行统计平均,提高精度。
\end{enumerate}


\section{对于问题一}
五、模型的建立与求解
5.1.1光的振动函数
时间变化的规律可以通过正弦函数和余弦函数来表述,其数学表达式为:


\end{document}

