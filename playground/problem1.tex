\section{对于问题一}
五、模型的建立与求解
5.1.1光的振动函数
时间变化的规律可以通过正弦函数和余弦函数来表述,其数学表达式为:


\subsection{问题一:平行平面薄膜双光束干涉的外延层厚度模型}

基于红外干涉法的原理,我们考虑外延层与衬底界面仅发生一次反射和透射的情形(如图1所示),建立确定碳化硅外延层厚度 \( d \) 的数学模型。模型的核心是计算入射红外光在空气-外延层和外延层-衬底界面的反射光束之间的光程差,并利用干涉极值条件反推 \( d \)。

\subsubsection{光程差与干涉条件}
入射光以角度 \( \theta \) 进入外延层,折射角 \( \phi = \arcsin(\sin \theta / n) \)。两束反射光的光程差为
\[
    \delta = 2 n d \cos \phi,
\]
其中 \( n \) 是外延层的折射率,\( d \) 是厚度。考虑相移:空气-外延层界面有半波损失(相移 \( \pi \)),外延层-衬底无(折射率相近)。因此,反射干涉的相长条件(反射率最大)为
\[
    \delta = \left( m + \frac{1}{2} \right) \lambda, \quad m = 0, 1, 2, \dots,
\]
相消条件(反射率最小)为 \( \delta = m \lambda \)。

\subsubsection{反射率计算}
使用菲涅尔公式计算反射率 \( R \)。对于 s 偏振斜入射,反射系数 \( r_s = \frac{n \cos \phi - \cos \theta}{n \cos \phi + \cos \theta} \),\( R = |r_s|^2 \)。类似地处理 p 偏振(假设非偏振光,取平均)。

\subsubsection{厚度求解模型}
从干涉条件,反推厚度
\[
    d = \frac{\left( m + \frac{1}{2} \right) \lambda}{2 n \cos \phi}.
\]
实际计算中,从反射谱 \( R(\nu) \) 提取极值波长 \( \lambda_k \),拟合 m-k 关系,求 \( d \)。假设 \( n = 2.55 \)(常数),模型输入为 \( \nu, R, \theta \),输出为 \( d \)(单位 μm)。

\begin{table}[htbp]
    \centering
    \caption{符号说明(扩展自团队定义)}
    \begin{tabular}{cccc}
        \hline
        符号   & 含义  & 单位       & 说明                              \\
        \hline
        \phi & 折射角 & ^{\circ} & \phi = \arcsin(\sin \theta / n) \\
        \hline
    \end{tabular}
\end{table}
