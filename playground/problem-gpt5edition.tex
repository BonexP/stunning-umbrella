% !TEX program = xelatex
\documentclass{ctexart}

\usepackage{amsmath,amssymb}
\usepackage[a4paper,margin=1in]{geometry}
\usepackage{hyperref}

\title{problem-gpt5edition(整理版)}
\author{ }
\date{\today}

\begin{document}
\maketitle

\section{平行平面薄膜干涉}
\subsection{问题设置与几何关系}
在折射率为 \(n_1\) 的均匀介质中放置一层厚度为 \(d\) 的平行平面薄膜,其折射率为 \(n_2\),并设 \(n_1<n_2\)。一束单色、平行入射的光 \(S\) 以入射角 \(\theta_1\) 入射到薄膜上表面。根据几何光学:
\begin{itemize}
    \item 一部分光在上表面(记为界面 \(M\),即 1--2 界面)反射,形成反射光线 a;
    \item 另一部分光折射进入薄膜,在下表面(记为界面 \(N\),即 2--3 界面)反射后再次折射出薄膜,形成光线 b。
\end{itemize}
两束出射光 a 和 b 近似平行,经透镜 \(L\) 聚焦在同一像点 \(P\)。

\subsection{相干性与光程差}
光线 a 与 b 源自同一入射光分束,满足相干叠加条件。在成像系统中可作辅助线保证比较同一路径的等效光程。透镜不引入附加光程差,可将两束光在 \(P\) 点的相位差归结为薄膜中往返段与空气段的光程差,并叠加界面处可能发生的“半波损失”(即 \(\pi\) 的相位突变)。
记薄膜内折射角为 \(\theta_2\),斯涅尔定律为
\[
    n_1\sin\theta_1=n_2\sin\theta_2,\quad \cos\theta_2=\sqrt{1-\left(\frac{n_1}{n_2}\right)^2\sin^2\theta_1}.
\]
则反射方向(两反射光束)在像点 \(P\) 的等效光程差为
\[
    \delta_{\rm ref} \,=\, 2\,n_2\,d\,\cos\theta_2 \, + \, \frac{\lambda}{2},
\]
其中 \(\tfrac{\lambda}{2}\) 来自一次从疏到密(\(n_1\to n_2\))界面反射引入的半波损失。用 \(\theta_1\) 消去 \(\theta_2\) 可写成
\[
    \delta_{\rm ref} \,=\, 2\,d\,\sqrt{\,n_2^2 - n_1^2\sin^2\theta_1\,}\; +\; \frac{\lambda}{2}.
\]
对于透射方向(两透射光束),反射过程不额外引入半波损失,其光程差为
\[
    \delta_{\rm tr} \,=\, 2\,n_2\,d\,\cos\theta_2 \,=\, 2\,d\,\sqrt{\,n_2^2 - n_1^2\sin^2\theta_1\,}.
\]

\subsection{垂直入射的特例}
当 \(\theta_1=0\)(法向入射)时,\(\theta_2=0\)、\(\cos\theta_2=1\),上式化为
\[
    \delta_{\rm ref}=2n_2d+\frac{\lambda}{2},\qquad \delta_{\rm tr}=2n_2d.
\]

\section{反射率与折射率(菲涅尔公式,重述版)}
设三介质的折射率分别为:入射侧(空气)\(n_1\)、薄膜(外延层)\(n_2\)、衬底 \(n_3\)。入射角为 \(\theta_1\),薄膜内折射角为 \(\theta_2\),衬底内折射角为 \(\theta_3\)。斯涅尔定律分别为
\[
    n_1\sin\theta_1=n_2\sin\theta_2,\qquad n_2\sin\theta_2=n_3\sin\theta_3.
\]
\subsection{垂直入射(\(\theta_1=0^\circ\))}
1--2 界面反射率
\[
    R_{12}=\left(\frac{n_1-n_2}{n_1+n_2}\right)^2;\qquad
    R_{23}=\left(\frac{n_2-n_3}{n_2+n_3}\right)^2.
\]
\subsection{斜入射(\(\theta_1>0^\circ\)),需按偏振分量区分}
\paragraph{s 偏振(电矢量垂直入射面)}
\[
    R_{s,12}=\left(\frac{n_1\cos\theta_1 - n_2\cos\theta_2}{n_1\cos\theta_1 + n_2\cos\theta_2}\right)^2
    =\left(\frac{n_1\cos\theta_1 - n_2\sqrt{1-(\tfrac{n_1}{n_2}\sin\theta_1)^2}}{n_1\cos\theta_1 + n_2\sqrt{1-(\tfrac{n_1}{n_2}\sin\theta_1)^2}}\right)^2,
\]
\[
    R_{s,23}=\left(\frac{n_2\cos\theta_2 - n_3\cos\theta_3}{n_2\cos\theta_2 + n_3\cos\theta_3}\right)^2
    =\left(\frac{n_2\cos\theta_2 - n_3\sqrt{1-(\tfrac{n_2}{n_3}\sin\theta_2)^2}}{n_2\cos\theta_2 + n_3\sqrt{1-(\tfrac{n_2}{n_3}\sin\theta_2)^2}}\right)^2.
\]
\paragraph{p 偏振(电矢量平行入射面)}
\[
    R_{p,12}=\left(\frac{n_1\cos\theta_2 - n_2\cos\theta_1}{n_1\cos\theta_2 + n_2\cos\theta_1}\right)^2
    =\left(\frac{n_1\sqrt{1-(\tfrac{n_1}{n_2}\sin\theta_1)^2} - n_2\cos\theta_1}{n_1\sqrt{1-(\tfrac{n_1}{n_2}\sin\theta_1)^2} + n_2\cos\theta_1}\right)^2,
\]
\[
    R_{p,23}=\left(\frac{n_2\cos\theta_3 - n_3\cos\theta_2}{n_2\cos\theta_3 + n_3\cos\theta_2}\right)^2
    =\left(\frac{n_2\sqrt{1-(\tfrac{n_2}{n_3}\sin\theta_2)^2} - n_3\cos\theta_2}{n_2\sqrt{1-(\tfrac{n_2}{n_3}\sin\theta_2)^2} + n_3\cos\theta_2}\right)^2.
\]

\section{问题一的数学模型(整理并补充)}
\subsection{基本量与约定}
\begin{itemize}
    \item 薄膜厚度:\(d\)(原文 \(e\))。
    \item 波数:\(\nu\)(cm\(^{-1}\)),与波长 \(\lambda\) 的关系为 \(\nu=1/\lambda\)、\(\lambda=1/\nu\)。
    \item 入射角:\(\theta_1\in\{10^\circ,\,15^\circ\}\)。
    \item 折射角:\(\theta_2\) 满足 \(n_1\sin\theta_1=n_2\sin\theta_2\),且 \(\cos\theta_2=\sqrt{1-(\tfrac{n_1}{n_2})^2\sin^2\theta_1}\)。
    \item 题设近似(同质外延):在碳化硅样品上采用 \(n_2=n_3=2.55\) 常数近似(问题一的简化)。
\end{itemize}
\subsection{反射/透射的光程差}
\[
    \delta_{\rm ref}=2\,n_2\,d\,\cos\theta_2+\frac{\lambda}{2},\qquad
    \delta_{\rm tr}=2\,n_2\,d\,\cos\theta_2.
\]
\subsection{干涉极值条件与“波数域”线性化}
薄膜双光束干涉在反射方向的极值条件可统一写成 \(\delta_{\rm ref}=m\,\lambda\ (m\in\mathbb{Z})\)。代入 \(\delta_{\rm ref}\) 得
\[
    2\,n_2\,d\,\cos\theta_2 \,=\, \Big(m-\tfrac{1}{2}\Big)\lambda.
\]
两边同乘以 \(\nu=1/\lambda\),得到“波数域”的线性关系
\[
    2\,n_2\,d\,\cos\theta_2\,\nu \,=\, m-\tfrac{1}{2}.
\]
可见,相邻同类型极值点(同为峰或同为谷)在波数轴上的间隔为常数
\[
    \Delta \nu \,=\, \nu_{k+1}-\nu_k \,=\, \frac{1}{2\,n_2\,d\,\cos\theta_2}.
\]
因而得到不依赖 \(m\) 的厚度计算公式
\[
    \boxed{\,d \,=\, \frac{1}{2\,n_2\,\cos\theta_2\,\Delta \nu}\,},\qquad
    \cos\theta_2=\sqrt{1-\left(\frac{n_1}{n_2}\right)^2\sin^2\theta_1}.
\]
单位换算:若 \(\nu\) 用 cm\(^{-1}\),则上式求得的 \(d\) 为 cm;换算为 \(\mu\)m:
\[
    \boxed{\,d_{\mu{\rm m}} \,=\, \frac{10^4}{2\,n_2\,\cos\theta_2\,\Delta \nu}\,}.
\]
注:对于透射方向,同理可得 \(2\,n_2\,d\,\cos\theta_2\,\nu=m\),相邻极值的 \(\Delta\nu\) 仍满足同一厚度公式,因此用“相邻同类型极值的波数间隔”估计厚度在反射与透射两种读法下是一致的。

\subsection{与反射率公式的衔接}
上述厚度公式基于几何相位条件,与菲涅尔反射率 \(R\) 的作用是相互补充的:菲涅尔公式给出每个界面处的幅度反射/透射比例(决定条纹对比度与可见性),而条纹周期(从而 \(\Delta\nu\))由相位项 \(2n_2 d \cos\theta_2\) 决定。问题一仅考虑一次反射/透射的双光束情形,利用极值间距即可稳健求 \(d\)。

\section{可执行的厚度求解步骤}
\begin{enumerate}
    \item 已知参量与角度修正:设定 \(n_1=1.0\)(空气),取题设近似 \(n_2=2.55\)。由入射角 \(\theta_1 \in \{10^\circ, 15^\circ\}\) 计算
          \[
              \cos\theta_2=\sqrt{1-\left(\frac{n_1}{n_2}\right)^2\sin^2\theta_1}.
          \]
    \item 从反射率–波数曲线中取极值:对测得的 \((\nu, R)\) 光谱进行平滑与极值检测,得到一串相邻的“同类型”极值波数 \(\{\nu_k\}\)。
    \item 求平均条纹间距:计算差分 \(\Delta \nu_k=\nu_{k+1}-\nu_k\),对有效区间取加权或简单平均 \(\overline{\Delta\nu}\)(可去除异常值)。
    \item 代入厚度公式:计算 \(d_{\mu{\rm m}}=\dfrac{10^4}{2\,n_2\,\cos\theta_2\,\overline{\Delta\nu}}\)。若有两组入射角数据,可分别计算后做一致性比较。
\end{enumerate}

\section{符号与单位说明(本段中出现的新记号)}
\begin{itemize}
    \item \(d\):外延层厚度,cm 或 \(\mu\)m。
    \item \(\lambda\):波长,cm 或 \(\mu\)m。
    \item \(\nu\):波数,cm\(^{-1}\),满足 \(\nu=1/\lambda\)。
    \item \(\theta_1\):入射角(在介质 \(n_1\) 中)。
    \item \(\theta_2\):薄膜内折射角,\(\cos\theta_2=\sqrt{1-(\tfrac{n_1}{n_2})^2\sin^2\theta_1}\)。
    \item \(n_1,n_2,n_3\):分别为空气、薄膜(外延层)、衬底的折射率。
    \item \(\delta_{\rm ref},\delta_{\rm tr}\):反射/透射两束干涉的光程差。
    \item \(R_{(\cdot)}\):各界面、各偏振分量的反射率。
\end{itemize}

\end{document}
