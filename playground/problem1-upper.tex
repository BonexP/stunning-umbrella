% !TEX program = xelatex

\documentclass{ctexart}
\usepackage{amsmath}
\usepackage{graphicx}

\begin{document}

\section{模型的建立与求解}

\subsection{问题一:平行平面薄膜双光束干涉的外延层厚度模型}

\subsubsection{光的振动函数}
时间变化的规律可以通过正弦函数和余弦函数来表述,其数学表达式为:

\[\overrightarrow{x}(t) = Acos(\omega t + \varphi)\]

\(\overrightarrow{x}(t)\):表示振动质点相对于其平衡位置的偏移量,用于描述质点在时刻 t 的位置。

\(A\) :光矢量的大小表示振动质点偏离其平衡位置的最大距离,以米为单位,反映了振动的强度。

\(\omega\) :角频率用于表征振动的速率,其单位为弧度/秒。它与频率 \(\nu\) 周期 \(T\) 的关系为 \(\omega = 2\pi\nu = \frac{2\pi}{T}\)。

\(\varphi\) :初相位是指在t=0时刻的相位,单位为弧度,用于确定振动的初始条件。

\subsubsection{光的振动描述——旋转矢量法}
如图1所示,采用旋转矢量法表示,一矢量绕原点O以角速度$\omega$逆时针匀速旋转,其瞬时在x轴上的投影即为光振动的简谐运动方程。

\begin{figure}[ht]
    \centering
    \fbox{\rule{2cm}{0pt} \rule{0pt}{2cm}} % 占位符用于图像
    \caption{旋转矢量法表示光的振动方程}
    \label{fig:1}
\end{figure}

如图1所示,该图通过旋转矢量法(亦称相量法)阐述了简谐振动的两种等效表示。左侧为几何表示:一振幅为$A$的矢量绕原点O以角频率$\omega$逆时针匀速旋转,其瞬时相位为$\phi=\omega t$。该矢量在纵轴(x轴)上的投影,即代表了振动系统在该时刻的瞬时位移x右侧为对应的时域表示,描述了上述投影随时间$t$的变化关系。如图所示,旋转矢量的运动精确地映射为一条以时间$t$为横轴、位移$x$为纵轴的正弦(或余弦)曲线,其数学表达为$x(t)=A\cos(\omega t+\varphi)$,与投影计算的结果完全一致。

旋转矢量法作为连接匀速圆周运动与一维简谐振动的桥梁,为理解振动参量提供了直观的几何图像:振幅$A$映射为旋转半径,决定了振动的强度;角频率 $\omega$映射为旋转角速度,决定了振动的快慢;初相位$\phi_0$映射为初始角位置,决定了振动的初始状态。这些几何关系清晰地解释了其时域波形$x(t)=A\cos(\omega t+\phi_0)$的特征。

\subsubsection{同方向同频率的光的合成}
考虑两个在同一直线上、具有相同角频率 $\omega$ 的简谐振动,其振动方程分别为:
$ \overrightarrow{x_1}(t) = A_1 \cos(\omega t + \phi_{1})$,其中 $A_1$ 是第一个振动的振幅, $\omega$ 是角频率, $\phi_{1}$ 是初相位;
$\overrightarrow{x_2}(t) = A_2 \cos(\omega t + \phi_{2})$,其中 $A_2$ 是第二个振动的振幅, $\phi_{2}$ 是初相位。合位移 $\overrightarrow{x}(t)$ 是两个分位移的矢量和,即 $\overrightarrow{x}(t) = \overrightarrow{x_1}(t) + \overrightarrow{x_2}(t)$ 。

把光矢量借助旋转矢量法表示在极坐标图上(图2),由余弦定理,理论推导可得,合振动也是简谐振动,表达式为 $\overrightarrow{x}(t) = A \cos(\omega t + \phi_0)$ ,其中:合振幅 $A = \sqrt{A_1^2 + A_2^2 + 2 A_1 A_2 \cos(\Delta \phi)}$ ,它由两个分振动的振幅 $A_1$ 、 $A_2$ 以及初相位差 $\Delta \phi = \phi_{2} - \phi_{1}$ 共同决定。

合初相位 \(\varphi:tan\varphi = \frac{A_{1}sin\varphi_{1} + A_{2}sin\varphi_{2}}{A_{1}cos\varphi_{1} + A_{2}cos\varphi_{2}}\) 。

\begin{figure}[ht]
    \centering
    \fbox{\rule{2cm}{0pt} \rule{0pt}{2cm}} % 占位符用于图像
    \caption{旋转矢量法合成同频率同方向的光}
    \label{fig:2}
\end{figure}

\subsubsection{光强的表示}
本文围绕光的干涉现象,阐述其波动叠加本质与基本规律,深入剖析相干条件的核心要素,并推导相干与非相干叠加场景下的光强计算方法及物理意义。


\begin{quote}
    \textbf{光矢量与光强}
\end{quote}

光矢量 $\vec{E}$ :光是电磁波,电场强度矢量 $\vec{E}$ 是光的振动矢量,称为光矢量,它的振动是光现象的主要体现。

光强 $I$ :光的强度(光强)与光矢量振幅 $A$ 的平方成正比,即 $I \propto A^2$ ,光强越大,干涉光越亮。

\begin{quote}
    \textbf{光的独立性与叠加原理}
\end{quote}


光的独立性原理:两列光在空间相遇时,各自的传播规律不受对方影响,继续保持原来的传播特性(如频率、波长、振动方向等)。

光的叠加原理:两列或多列光在空间某点相遇时,该点的光矢量是各列光在该点光矢量的矢量和。


\begin{quote}
    \textbf{相干条件}
\end{quote}
两束光的相干叠加(产生稳定干涉条纹)需满足以下条件:

    1.频率相同:\(\omega_{1} = \omega_{2}\)( $\omega$ 为角频率,频率 $\nu= \frac{\omega}{2\pi}$,频率相同意味着振动的"快慢"一致)。

    2.振动方向夹角稳定且非垂直:两列光的振动方向(光矢量方向)的夹角 $\theta$ 不随时间 $t$ 变化,且 $\theta \neq 90^\circ$ (若垂直,光矢量叠加时部分分量会抵消,难以形成稳定干涉)。

    3.相位差稳定:两列光的相位差 $\Delta \phi$ 不随着时间 $t$ 变化(相位差稳定才能保证叠加后光强的分布趋于稳定)。

\begin{quote}
    \textbf{光强的叠加}
\end{quote}

相干叠加:若两列光满足相干条件,叠加后的光强为 $I = I_1 + I_2 + 2 \sqrt{I_1 I_2} \cos(\Delta \phi)$ 。其中, $2 \sqrt{I_1 I_2} \cos(\Delta \phi)$ 是干涉项,它使光强分布随相位差 $\Delta \phi$ 变化:

当 $\Delta \phi = \pm 2k\pi$ 时, $\cos(\Delta \phi) = 1$ ,光强 $I = I_1 + I_2 + 2 \sqrt{I_1 I_2}$ ,达到相长干涉(光强最大)。

当 $\Delta \phi = \pm (2k+1)\pi$ 时, $\cos(\Delta \phi) = -1$ ,光强 $I = I_1 + I_2 - 2 \sqrt{I_1 I_2}$ ,达到相消干涉(光强最小,若 $I_1 = I_2$ ,则光强为 $0$ )。

\subsubsection{光程和光差}
设某一频率为 $f$的单色光在真空中的传播速度为 $c$,波长为 $\lambda$。当该光在折射率为 $n$ 的介质中传播时,其速度变为 $v$,波长变为 $\lambda'$。

\[\lambda_{n} = \frac{u}{v} = \frac{c/n}{v} = \frac{\lambda}{n}\]

上述公式表明,特定频率的光在折射率为 $n$ 的介质中传播时,其波长为真空中的波长的 $1/n$ 倍。根据波动理论,当每束光从光源传播至相遇点经过 1 个单位距离后,其相位变化量为


\[\Delta\phi = 2\pi\frac{l}{\lambda}\]

由于同一频率的光在不同介质中的波长各不相同,因此上述公式中的 $\lambda'$ 应该理解为光在相应介质中的波长。因此,当单色光在折射率为 $n$的介质中传播一定距离后,其相位变化量为

\[\Delta\phi = 2\pi\frac{l}{\lambda_{n}} = 2\pi\frac{nl}{\lambda}\]


上述公式表明,光在折射率为 $n$的介质中传播一定距离 $d$后,其相位变化量与光在真空中传播相同距离时的相位变化量是相等的。因此,我们将光在介质中传播的距离 $d$与该介质的折射率 $n$的乘积 $n\cdot d$ 称为光程。

\subsubsection{光程差与干涉的关系}

如图 3 所示,若两个初相均为 $\phi_0$ 的相干光源 $S_1$ 和 $S_2$ 发出的光在 P 点相遇,则它们在 P 点的相位差为

\[\Delta\phi = \left( \phi - 2\pi\frac{n_{2}r_{2}}{\lambda} \right) - \left( \phi - 2\pi\frac{n_{1}r_{1}}{\lambda} \right) = \frac{2\pi}{\lambda}\left( n_{1}r_{1} - n_{2}r_{2} \right)\]

\begin{figure}[ht]
    \centering
    \fbox{\rule{2cm}{0pt} \rule{0pt}{2cm}} % 占位符用于图像
    \caption{计算相干光的光程差}
    \label{fig:3}
\end{figure}
令 $\delta = n_2 r_2 - n_1 r_1$ 称为两束光的光程差,其中$n_1$和$n_2$是两种介质的折射率,则上式可写为


\[\Delta\phi = \frac{2 \pi\delta }{\lambda}\]


因此,在波动光学中,干涉相长和干涉相消的条件可以通过光程差来进行表述

\[\delta = \pm \text{kλ}(k = 0,1,2,\cdots)\text{~}\text{干涉相长}\text{\ (}\text{明纹}\text{)}\]

\[\delta = \pm (2k + 1)\lambda/2(k = 0,1,2,\cdots)\text{~}\text{干涉相消}\text{\ (}\text{暗纹}\text{)}\]

\end{document}
