% !TEX program = xelatex

\documentclass{ctexart}
\usepackage{amsmath}
\usepackage{graphicx}

\begin{document}

\section{模型的建立与求解}

\subsection{问题一:平行平面薄膜双光束干涉的外延层厚度模型}

\subsubsection{光的振动函数}
时间变化的规律可以通过正弦函数和余弦函数来表述,其数学表达式为:x ⃗(t)=Acos⁡(ωt+φ)
:表示振动质点相对于其平衡位置的偏移量,用于描述质点在时刻 $t$ 的位置。
:光矢量的大小表示振动质点偏离其平衡位置的最大距离,以米为单位,反映了振动的强度。
:角频率用于表征振动的速率,其单位为弧度/秒。它与频率  周期  的关系为 。
:初相位是指在$t=0$时刻的相位,单位为弧度,用于确定振动的初始条件。

\subsubsection{光的振动描述——旋转矢量法}
如图1所示,采用旋转矢量法表示,一矢量绕原点O以角速度$\omega$逆时针匀速旋转,其瞬时在x轴上的投影即为光振动的简谐运动方程。

\begin{figure}[ht]
    \centering
    \fbox{\rule{2cm}{0pt} \rule{0pt}{2cm}} % 占位符用于图像
    \caption{旋转矢量法表示光的振动方程}
    \label{fig:1}
\end{figure}

如图1所示,该图通过旋转矢量法(亦称相量法)阐述了简谐振动的两种等效表示。左侧为几何表示:一振幅为$A$的矢量绕原点O以角频率$\omega$逆时针匀速旋转,其瞬时相位为$\phi=\omega t$。该矢量在纵轴(x轴)上的投影,即代表了振动系统在该时刻的瞬时位移x右侧为对应的时域表示,描述了上述投影随时间$t$的变化关系。如图所示,旋转矢量的运动精确地映射为一条以时间$t$为横轴、位移$x$为纵轴的正弦(或余弦)曲线,其数学表达为)$x(t)=A\cos(\omega t+)$,与投影计算的结果完全一致。
旋转矢量法作为连接匀速圆周运动与一维简谐振动的桥梁,为理解振动参量提供了直观的几何图像:振幅$A$映射为旋转半径,决定了振动的强度;角频率 $\omega$映射为旋转角速度,决定了振动的快慢;初相位$\phi_0$映射为初始角位置,决定了振动的初始状态。这些几何关系清晰地解释了其时域波形$x(t)=A\cos(\omega t+\phi_0)$的特征。

\subsubsection{同方向同频率的光的合成}
考虑两个在同一直线上、具有相同角频率 $\omega$ 的简谐振动,其振动方程分别为:
$x_1(t) = A_1 \cos(\omega t + \phi_{10})$,其中 $A_1$ 是第一个振动的振幅, $\omega$ 是角频率, $\phi_{10}$ 是初相位;
$x_2(t) = A_2 \cos(\omega t + \phi_{20})$,其中 $A_2$ 是第二个振动的振幅, $\phi_{20}$ 是初相位。合位移 $x(t)$ 是两个分位移的矢量和,即 $x(t) = x_1(t) + x_2(t)$ 。
把光矢量借助旋转矢量法表示在极坐标图上(图2),由余弦定理,理论推导可得,合振动也是简谐振动,表达式为 $x(t) = A \cos(\omega t + \phi_0)$ ,其中:合振幅 $A = \sqrt{A_1^2 + A_2^2 + 2 A_1 A_2 \cos(\Delta \phi)}$ ,它由两个分振动的振幅 $A_1$ 、 $A_2$ 以及初相位差 $\Delta \phi = \phi_{20} - \phi_{10}$ 共同决定。合初相位 $\phi_0 = \arctan \left( \frac{A_1 \sin \phi_{10} + A_2 \sin \phi_{20}}{A_1 \cos \phi_{10} + A_2 \cos \phi_{20}} \right)$ 。

\begin{figure}[ht]
    \centering
    \fbox{\rule{2cm}{0pt} \rule{0pt}{2cm}} % 占位符用于图像
    \caption{旋转矢量法合成同频率同方向的光}
    \label{fig:2}
\end{figure}

\subsubsection{光强的表示}
本文围绕光的干涉现象,阐述其波动叠加本质与基本规律,深入剖析相干条件的核心要素,并推导相干与非相干叠加场景下的光强计算方法及物理意义。

\paragraph{光矢量与光强}
光矢量 $\vec{E}$ :光是电磁波,电场强度矢量 $\vec{E}$ 是光的振动矢量,称为光矢量,它的振动是光现象的主要体现。
光强 $I$ :光的强度(光强)与光矢量振幅 $A$ 的平方成正比,即 $I \propto A^2$ ,光强越大,干涉光越亮。

\paragraph{光的独立性与叠加原理}
光的独立性原理:两列光在空间相遇时,各自的传播规律不受对方影响,继续保持原来的传播特性(如频率、波长、振动方向等)。光的叠加原理:两列或多列光在空间某点相遇时,该点的光矢量是各列光在该点光矢量的矢量和。

\paragraph{相干条件}
两束光的相干叠加(产生稳定干涉条纹)需满足以下条件:
频率相同:( $\omega$ 为角频率,频率 $f$ ,频率相同意味着振动的"快慢"一致)。
2.振动方向夹角稳定且非垂直:两列光的振动方向(光矢量方向)的夹角 $\theta$ 不随时间 $t$ 变化,且 $\theta \neq 90^\circ$ (若垂直,光矢量叠加时部分分量会抵消,难以形成稳定干涉)。
3.相位差稳定:两列光的相位差 $\Delta \phi$ 不随着时间 $t$ 变化(相位差稳定才能保证叠加后光强的分布趋于稳定)。

\paragraph{光强的叠加}
相干叠加:若两列光满足相干条件,叠加后的光强为 $I = I_1 + I_2 + 2 \sqrt{I_1 I_2} \cos(\Delta \phi)$ 。其中, $2 \sqrt{I_1 I_2} \cos(\Delta \phi)$ 是干涉项,它使光强分布随相位差 $\Delta \phi$ 变化:
当 $\Delta \phi = 2k\pi$ 时, $\cos(\Delta \phi) = 1$ ,光强 $I = I_1 + I_2 + 2 \sqrt{I_1 I_2}$ ,达到相长干涉(光强最大)。
当 $\Delta \phi = (2k+1)\pi$ 时, $\cos(\Delta \phi) = -1$ ,光强 $I = I_1 + I_2 - 2 \sqrt{I_1 I_2}$ ,达到相消干涉(光强最小,若 $I_1 = I_2$ ,则光强为 $0$ )。

\subsubsection{光程和光差}
设某一频率为 $f$的单色光在真空中的传播速度为 $c$,波长为 $\lambda$。当该光在折射率为 $n$ 的介质中传播时,其速度变为 $v$,波长变为 $\lambda'$。
上述公式表明,特定频率的光在折射率为 $n$ 的介质中传播时,其波长为真空中的波长的 $1/n$ 倍。根据波动理论,当每束光从光源传播至相遇点经过 1 个单位距离后,其相位变化量为
由于同一频率的光在不同介质中的波长各不相同,因此上述公式中的 $\lambda'$ 应该理解为光在相应介质中的波长。因此,当单色光在折射率为 $n$的介质中传播一定距离后,其相位变化量为
上述公式表明,光在折射率为 $n$的介质中传播一定距离 $d$后,其相位变化量与光在真空中传播相同距离时的相位变化量是相等的。因此,我们将光在介质中传播的距离 $d$与该介质的折射率 $n$的乘积 $n\cdot d$ 称为光程。

\subsubsection{光程差与干涉的关系}
如图 3 所示,若两个初相均为 $\phi_0$ 的相干光源 $S_1$ 和 $S_2$ 发出的光在 P 点相遇,则它们在 P 点的相位差为
\begin{figure}[ht]
    \centering
    \fbox{\rule{2cm}{0pt} \rule{0pt}{2cm}} % 占位符用于图像
    \caption{计算相干光的光程差}
    \label{fig:3}
\end{figure}
令 $\delta = n_2 r_2 - n_1 r_1$ 称为两束光的光程差,其中$n_1$和$n_2$是两种介质的折射率,则上式可写为
因此,在波动光学中,干涉相长和干涉相消的条件可以通过光程差来进行表述

\end{document}
